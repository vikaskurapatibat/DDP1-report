\chapter{Results and Discussions}
The following test case was implemented in PySPH using the method of Interface curvature described in the above sections. \\


\section{Stability of an Interface: Problem Statement}

Previous studies using the CSF formulation have reported a numerical instability at the fluid-fluid interface. \cite{rudman, Surface}. This instability leads to parasitic currents(non-zero velocities) at the interfaces. Using SPH, these currents could lead to particle disorder at the fluid-fluid interface, showing the effect of the diffusing at the interface. \\

This phenomena is studied here: A simple static case in a periodic domain spanning 0.5 units in x-directoin and 1 unit in y-direction with upper half with fluid of color 1 and lower half with fluid of color 0 is simulated. The fluid densities were considered constant of 1, the surface tension coeffient to be 1 and inviscid flow is considered. The speed of sound for this simulation is set to 20 untis.\cite{Morris} At t=0, a constant velocity in both directions is set so as the KE of each particle is 6 orders less than their internal energy.  The evolution of Kinetic Energy with time is recorded for different SPH parameters like smoothing length and spline. The smoothing length hdx = 1.0 and 1.5 are considered and varied smoothing length for different equations is considered where hdx = 1.0 is used all through except when calculating the surface delta function. In this way, by using a larger smoothing length for curvature calculation, more reliable estimates of curvature are obtained in the region where surface normals are non-zero.

\section{Results}

The following is the plot of KE vs time and against the results extracted from the paper \cite{Morris}. The discrepancy in the exact values is because of lack of initialization data in the paper like initial velocities' distribution, method of calculating internal energy etc., But it can be noted that the trend and the approximate values are in agreement from the plot below.

\begin{figure}[H]
\centering
\includegraphics[width = \textwidth]{../../case1.png}
\caption{Evolution of Kinetic Energy with time of the instability}
\end{figure}


\section{Discussions}

It can be seen from the plot that the variableh case has better stability of the interface preventing much diffusion of the particles into each other.\\

It is seen that, for the same parameters, the quintic spline is more stable than the cubic spline.\\

Also, the increased smoothing length caused to a considerable amount of stability in the interface. The best stability properties are however exhibited when the smoothing length for curvature is higher than that of delta function calculation.

