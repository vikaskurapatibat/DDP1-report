%%% Time-stamp: <mainrep.tex 19:57, 17 Jul 2016 by P Sunthar>
%%% $Log:$
% This document describes how to use iitbreport style
%********************************************************************

%\documentclass[11pt,a4paper,openright]{report}
\documentclass[twoside]{iitbreport}
% \documentclass[oneside]{iitbreport}


%% Default spacing: 1.5
%% Default font size: 12pt
%% Default font: txfonts (similar to times new roman) 

%% Selectively comment out sections that you want to be left out but
%% maintaining the page numbers and other \ref
\includeonly{%
  intro/introduction,
  lit/literature,
  expt/experimental,
  rnd/results, 
  dec,abs,pub,ack
}

%%% Some commonly used packages (make sure your LaTeX installation
%%% contains these packages, if not ask your senior to help installing
%%% the packages)

\usepackage{booktabs}
\usepackage{graphicx}
\usepackage{caption}
\usepackage{subcaption}
\usepackage{float}
\usepackage{hyperref}
\graphicspath{{expt/}}


%%% Macro definitions for Commonly used symbols
\newcommand{\Rey}{\ensuremath{\mathrm{Re}}}
\newcommand{\avg}[1]{\ensuremath{\overline{#1}}}
\newcommand{\tenpow}[1]{\ensuremath{\times 10^{#1}}}
\newcommand{\pder}[2]{\ensuremath{\frac{\partial#1}{\partial#2}}}

% Referencing macros
\newcommand{\Eqref}[1]{Equation~\eqref{#1}}
\newcommand{\Tabref}[1]{Table~\ref{#1}}
\newcommand{\Figref}[1]{Figure~\ref{#1}}
\newcommand{\Appref}[1]{Appendix~\ref{#1}}


\begin{document}
	
%%********************************Frontmatter***********************
% In frontmatter everything comes with roman numbering	
\pagenumbering{roman}
\setcounter{page}{1}

%*******************************************************************
%                         Title Page                            
%*******************************************************************
\title{Simulation of Multiphase Flows using Smooth Particle Hydrodynamics(SPH)}
\author{Vikas Kurapati}

%% Print the date. Today's date comes by default, change it here to 
%% other date format, if required:

\date{\today}
%\date{10 Mar 2016}


%% The type of the report can be set here

% \reporttype{A Seminar Report}
%\reporttype{A Thesis}
%\reporttype{A Dissertation}
\reporttype{A Project Report}

%% Name of the degree
% \degree{Doctor of Philosophy}
\degree{Master of Technology}


%% Department/Centre Name
\dept{Department of Aerospace Engineering}

%% Supervisor and cosupervisor/excosupervisor are not essential parts
%% of a report title page, as it is your report!

%% But if you **have** to put it uncomment these
\supervisor{Prof. Prabhu Ramachandran}
%\cosupervisor{Co-super name}
%\excosupervisor{External Supervisor}

%% Roll number
\rollnum{Roll No. 130010058}

\maketitle

%*******************************************************************
%                         Copyright Page                          
%******************************************************************* 
%\mycopyright                    

%*******************************************************************
%                         Dedication Page                         
%*******************************************************************
% \dedication[Dedicated to \ldots]        
%\addintoc{Dedication}

%*******************************************************************
%                        Certificate Page                         
%*******************************************************************
%\makecertificate[change title name]{report type} 
% \makecertificate{seminar report} 
%\makecertificate{thesis}
%\makecertificate{dissertation}
\makecertificate{project report}

%\addintoc{Certificate}

%*******************************************************************
%                         Approval Sheet                         
%*******************************************************************
%\makeapproval{thesis}
%\makeapproval{dissertation}

%*******************************************************************
%                          Declaration                           
%*******************************************************************
%==================================dec.tex================================
%
\begin{Declaration}
\noindent
I declare that this written submission represents my ideas in my own words and where others' ideas or words have been included, I have adequately cited and referenced the original sources. I declare that I have properly and accurately acknowledged all sources used in the production of this report. I also declare that I have adhered to all principles of academic honesty and integrity and have not misrepresented or fabricated or falsified any idea/data/fact/source in my submission. I understand that any violation of the above will be a cause for disciplinary action by the Institute and can also evoke penal action from the sources which have thus not been properly cited or from whom proper permission has not been taken when needed.

%
%
%
%
%
%
%

\DecSign[\today]



%
\end{Declaration}
%========================================================================
















 
%\addintoc{Declaration}

%%%
\acknowledgments

I would like to thank Prof. Prabhu Ramachandran for his expert guidance, motivation and for his encouragment throughout the course of this project.
I also would like to thank him for creating interest in me through the course of Particle Methods.\\
\noindent
I would like to thank my dear friend Suraj and my DDP group members Dinesh and Abhinav for clearing doubts when required and for keeping me motivated in this  project.\\
\noindent
I would like to thank my friend Mrinal for introducing me to this concept of Particle Methods and for pushing me to pick up a project in this field.






\signature{\today}
%\signature[Indian Institute of Technology Bombay]{\today}

%========================================================================

%%% Local Variables: 
%%% mode: latex
%%% TeX-master: "../mainrep"
%%% End:            


%******************************************************************
%                          Abstract                             
%******************************************************************  
%============================= abs.tex================================
\begin{Abstract}

Multiphase flows are encountered often in day-to-day life like mixing of two liquids etc., For many such flows, the interaction at the surface is important. In this work, multiphase flows' interaction is studies with emphasis of surface tension in SPH.\\

Smoothed-Particle Hydrodynamics(SPH) is a mesh-free Lagrangian particle method which is now used in various fields like astrophysics, fluid mechanics and solid dynamics. This SPH method has been used to solve the problems of surface tension. To do this, the important issue to be addressed is the calculation of interface accurately. Grid based schemes which interfacial tracking or interface capturing techniques cause issues in mass conservation, curvature capturing and interface diffusion. So, this work tries to overcome these using a grid-free scheme in Smoothed-Particle Hydronamics.

%
%
%
%
%
\end{Abstract}
%=======================================================================

                    

%******************************************************************
%                         Contents list                         
%******************************************************************
\figurespagefalse
\tablespagefalse
\makecontents % Creats toc, lof, and lot

%******************************************************************
%                        Notations                              
%******************************************************************
\notations[4cm]{List of Symbols}      

%%********************************Mainmatter***********************
% In mainmatter everything comes with arabic numbering	
\cleardoublepage
\setcounter{page}{1}
\pagenumbering{arabic}

%******************************************************************
%                         Chapters                           
%****************************************************************** 

\newcommand{\etas}{\ensuremath{\eta_{\mathrm{s}}}}


\chapter{Introduction}

The three main forces involved in multi phase flows are the pressure forces, viscous forces and surface tension foces. Many techniques have been developed to simulate multiphase flows with surface tension. Smoothed Particle Hydrodynamics is a computational technique which was originally developed to model astrophysical problems \cite{Monaghan1977}, which was later extended to model a wide variety of problems in compuational physics. \cite{Monaghan1992, Monaghan1994}. SPH is a fully Lagrangian technique where fluid interfaces are advected with very little numerical diffusion. The SPH formalism readily accomodates extra physical effects and highly irregular, mobile or even deformable boundaries. However, SPH can often be more computationally expensive than competing methods for idealized problems. In this work we simulate surface tension acting on an interface between two fluids. 

\section{Methodology}

The SPH method was introduced in the late 1970's to solve the problems in astrophysics. There is no need of underlying grid for spatial discretization. As the SPh simulation is based on Lagrangian frame of reference, the fluid is discretized by moval spatial discretization points called particles. Only the initial domain is to be discretized and then these particles are moved according to their interactions. The physical propeerties like mass, density, velocity and internal energy are assigned to these particles. In accordance with the governing conservation equations, the fluid characteristics are calculated at each particle.



\chapter{Basics of SPH Formulations}

In the following sections, the derivation and formulations of SPH methodology will be explained.
In SPH, we interpolate any function to be expressed in terms of its values at a set of disordered points called as the particles.
Every physical property of a fluid is regarded as a spatial function f(r). 
The ideas are given in \cite{Monaghan1977} and \cite{Lucy}. 
Each function in space be exactly represented by a convolution of the function f(r) itself with the Diract function $\delta (r)$.

\begin{equation}
 f(r) = \int f(r')\delta(r - r')dr'
\end{equation}

In SPH, the Dirac function $\delta (r)$ is replaced by a so called interpolating kernel function W(r-r', h), which resembles a Gauss distribution with compact support. Hence, in SPH, the integral approximation of any function f(r) is defined by

\begin{equation}
 F_1 (r) = \int F(r') W(r-r', h) dr' ,
\end{equation}

\noindent
where the integration is over the entire space, and W is an interpolating kernel which has the following two properties.

\begin{equation}
 \int W(r-r', h)dr' = 1
\end{equation}

\noindent
and 

\begin{equation}
 \lim_{h\to0} W(r-r', h) = \delta(r-r')
\end{equation}

\noindent
where the limited is to be interpreted as the limit of the corresponding integral interpolants. The smoothing length 'h', determines the radius of influence of the kernel function. \\

For numerical work, the integral representation is approximated by a summation interpolant as below.

\begin{equation}
 F_s (r) = \sum_{j} F_j \frac{m_j}{\rho_j} W(r-r_j, h)
 \label{interpolation}
\end{equation}

\noindent
where the summation index j is used to denote a particle label, the summation is performed over all the particles. Particle j carries mass $m_j$, position $r_j$, density $\rho_j$ and velocity $v_j$. The value of any quantity A at position $r_j$ is written as $A_j$. \\

\noindent
And the differentials and gradients are written as 

\begin{equation}
 \nabla F(r) = \sum_j \frac{m_j}{\rho_j}F_j \nabla W(r-r_j, h)
  \label{gradinterpolation}
 \end{equation}


\section{SPH formulation of basic Navier Stokes Equations}

To simulate a fluid flow, the Navier Stokes Equations are to be 
formulated into SPH. For this, equations \ref{interpolation} and \ref{gradinterpolation}
are used. These give us the approximations for the differentials of density, velocity and energy.

\subsection{Continuity Equation}

Continuity Equation gives the evolution of density. The formulations
are as given below:

\begin{equation}
 \rho_i = \sum_j m_j W_{ij}
 \label{summation}
\end{equation}

This form of equation for density is called the Summation Density
which uses SPH formulation of \ref{interpolation}. \\

\noindent
Density can also be evolved by using $\frac{d\rho}{dt} + \nabla . \mathbf{v} = 0$

\begin{equation}
 \frac{d\rho_i}{dt} = \sum_j m_j \mathbf{v_{ij}} \nabla_i W_{ij}
  \label{Continuity}
 \end{equation}

where $\mathbf{v_{ij}} = \mathbf{v_i} - \mathbf{v_j}$.\\


\subsection{Momentum Equation}

The momentum Equation generally contains pressure and viscous terms which are discussed here.
Any extra forces like stress, surface tension forces are to be dealth
seperately and added to the momentum equation.

\subsubsection{Pressure Force}

The force per unit mass due to pressure is given as -$\frac{\nabla P}{\rho}$.
The gradient of the pressure can be estimated as:

\begin{equation}
 \rho_i \nabla P_i = \sum_j m_j (P_j - P_i) \nabla_i W_{ij}
\end{equation}

This equation gives zero forces for constant pressures but doesn't
conserve linear and angular momentum. Hence, a differen SPH formulation
is required.\\

For the same reason, a different formulation of the pressure gradient is used as follows:

\begin{equation}
 \frac{\nabla P}{\rho} = \nabla \left( \frac{P}{\rho}\right) + \frac{P}{\rho^2}\nabla P
\end{equation}

This when written in SPH form gives pressure to be as:

\begin{equation}
 \frac{d\mathbf{v_i^p}}{dt} = -\sum_j m_j \left( \frac{P_j}{\rho_j^2} + \frac{P_i}{\rho_i^2}\right)\nabla_i W_{ij}
\end{equation}

\noindent
This form of pressure force conserves momentum as it is asymmetric. 

\subsubsection{Viscous Force}

If there is no Physical Laminar viscosity in the flow, in general
an artificial viscosity is used in SPH to improve the numerical stability.\\

The commonly used artificial viscosity used term as given in \cite{Monaghan1983}:

\begin{equation}
 \frac{d\mathbf{v_i^\nu}}{dt} = - \sum_j m_j \Pi_{ij} \nabla_i W_{ij}
\end{equation}

where the viscosity term is:

\begin{equation}
 \Pi_{ij} = -\nu \left( \frac{min(\mathbf{v_{ij}}, \mathbf{x_{ij}}, 0)}{(\mathbf{x_{ij}}^2 + \epsilon h_{ij}^2)}\right), 
  \label{pi}
 \end{equation}

\noindent
where the viscous factor is:

\begin{equation}
 \nu = \frac{\alpha h_{ij}c_s}{\rho_{ij}}
\end{equation}

\noindent
where $\alpha$ is the viscosity constant, $c_s$ is the artificial
speed of sound, $\mathbf{v_{ij}} = \mathbf{v_i} - \mathbf{v_j}$
and $\mathbf{x_{ij}} = \mathbf{x_i} - \mathbf{x_j}$. In equation \ref{pi},
$\epsilon \sim 0.01$ is used to avoid blowing up when $x_{ij} = 0$ and average smoothing length is given by
$h_{ij} = 0.5*(h_i + h_j)$. This viscous force basically gives a repulsive force between approaching particles.

\subsection{The Energy Equation}

The equation for the rate of change of thermal energy per unit mass is given by:

\begin{equation}
 \frac{du}{dt} = -\left( \frac{P}{\rho} \right)\nabla. \mathbf{v}
\end{equation}

\noindent
which when written in SPH form gives:

\begin{equation}
 \frac{du_i}{dt} = \left(\frac{P_i}{\rho_i^2}\right)\sum_j m_j \mathbf{v_{ij}}.\nabla_iW_{ij}
\end{equation}

\noindent
or by transforming the equation to:

\begin{equation}
 \frac{du}{dt} = -\nabla\left(\frac{P\mathbf{v}}{\rho}\right) + \mathbf{v}.\nabla\left(\frac{P}{\rho}\right)
\end{equation}

\noindent 
the equation can be rewritten as:

\begin{equation}
 \frac{du_i}{dt} = \sum_j \left( \frac{P_j}{\rho_j^2} \right)\mathbf{v_ij}.\nabla_i W_{ij}
\end{equation}

\noindent
By taking the average of the above two representations, we find:

\begin{equation}
 \frac{du_i}{dt} = \frac{1}{2} \sum_j m_j \left( \frac{P_j}{\rho_j^2} + \frac{P_i}{\rho_j^2}\right)\mathbf{v_{ij}}.\nabla_i W_{ij}
\end{equation}

\noindent
note that this has symmetric factors.


\section{Moving the Particles}

There are two ways to move the particles. One way is:

\begin{equation}
 \frac{d\mathbf{r_i}}{dt} = \mathbf{v_i}
\end{equation}

\noindent
or the XSPH variant as given in \cite{Monaghan1983}:

\begin{equation}
\frac{d\mathbf{r_i}}{dt} = \hat {\mathbf{v_i}} = \mathbf{v_i} + \epsilon \sum_j m_j \left(\frac{\mathbf{v_{ji}}}{\rho_{ij}}\right) W_{ij}
\end{equation}

\noindent
where $\rho_{ij} = 0.5*(\rho_i + \rho_j)$ and $\epsilon (0 \leq \epsilon \leq 1)$ as a constant. 
This XSPH correction helps in moving a particles with a velocity that is closer to the
average velocity in its neighbourhood. XSPH is found to increase dispersion instead of introducin
dissipation. The XSPH correction is useful in simulating nearly incompressible fluids like water, which keeps 
particles' motion orderly in the absence of viscosity.

\section{Equation of State}

For a closed system of equations, we need an Equation of State(EOS) which evaluates pressure.
Generally, an equation of state provides a direct coupling of density and pressure. An incompressible fluid 
can be thought of as a weakly compressible fluid. This can be done by using a stiff EOS.
Hence, pressure variations are high for small variations of density also. A numerical speed of sound
$c_0 = \sqrt{\left(\frac{dP}{d\rho}\right)_s}$ is used in the EOS.\\

The Suitable equation as given by Tait is given by :

\begin{equation}
 \Delta P = \frac{c_0^2 \rho_0}{\gamma} \left[ \left( \frac{\rho}{\rho_0} \right)^{\gamma} - 1 \right]
\end{equation}

\noindent
Using this equation, the pressure can be computed and used in the momentum equations.



\chapter{Physical Foundations of Multiphase flows}

\section{Introduction}
 
Three main forces are considered in the regime of multiphase flows: Pressure, Viscous and Surface Tension forces. \\
\noindent
In the following sections we discuss the calculations of Surface tension forces. \\
\noindent
To calculate the surface tension force, we will need to calculate the interface curvature of different phases.

\section{Interface Tracking Techniques}

In the case of immiscible fluids, each corresponding to a differenct 'color', c, interface tracking can be achieved by simulating the advection of the color function. \cite{Morris}

\begin{equation}
\frac{\partial c}{\partial t} + \mathbf{v}.\nabla c = 0 
\end{equation}

The color function may be evolved in a Lagrangian fashion by assigning it as a physical property to 'particles', which are then advected through the computational domain. In genral, this may be achieved using particles on the interface alone(surface-marker methods) or by employing particles which fill the entire computational domain(volume-marker methods). Surface markers have been used extensively to track the location of interface with high accuracy. This approach may be exploited by fully Lagrangian Techniques like SPH. With SPH, each extra phase is modelled simply by introducing an extra species of a particle of different color property.

\section{The Continuum Surface Force Method}

The Continuum surface force(CSF) method \cite{Brackbill} permits numerical simulation of surface tension without placing restrictions upon the flow geometry. The CSF approach models process localized to a fluid interface by applying them to fluid elements in the transition region of the interface. Interfacial phenomena, such as surface tension and phase change, are translated into volume processes having a net effect that emulates the desired physics.
In the CSF model, surface tension is translated into a force per unti volume $\mathbf{F_s}$, by

\begin{equation}
 \mathbf{F_s} = \mathbf{f_s} \delta_s
\end{equation}

\noindent
where $\delta_s$ is a normalized surface delta function, which peaks at the interface and $\mathbf{f_s}$ is the force per unit area given by

\begin{equation}
 \mathbf{f_s} = \sigma \kappa \mathbf{\hat{n}} + \nabla_s \sigma
 \label{forceperarea}
\end{equation}

\noindent
where $\sigma$ is the surface tension coefficient, $\mathbf{\hat{n}}$ is the unit normal to the interface, $\kappa$ is the curvature to the interface and $\nabla_s$ is the surface gradient. The second term in \ref{forceperarea} acts tangentially to the interface, forcing fluid from regions of low surface tension to higher surface tension. In this work surface tension is assumed constant throughout the fluid and hence that part is neglected. The first term in \ref{forceperarea} acts normal to the interface corresponding to the net surface tension force due to the local curvature. This force acts to smooth regions of high curvature, in an attempt to reduce the total surface energy.
The normal in \ref{forceperarea} can be obtained using

\begin{equation}
 \mathbf{n} = \frac{\nabla c}{[c]}
\end{equation}

\noindent
where c is the color function identifying each fluid in the simultion and [c] is the jump in c across the interface. The curvature can be calculated as
\begin{equation}
 \kappa = \nabla . \mathbf{\hat{n}}
\end{equation}

\noindent
There are many possible choices for $\delta_s$ however, it should be normalized such that its integral through the boundary is one. This is necessary for the correct physics of the interface to be recovered as the resolution is increased. The funciton should also be non-zero only in those fluid elements that correspond to the transition regions in the numerical method. The surface delta function employed in this work is

\begin{equation}
 \delta_s = \left| \mathbf{n} \right |
\end{equation}


\chapter{SPH formulations}

Using SPH, the fluid is represented by particles, generally of fixed mass, which follow the fluid motion according to their governing differential equations. These governing equations become expressions for interparticle forces and fluxes when written in SPH form. Using the standard approach of SPH, the particles move with the local fluid velocity carrying a mass m. Each particle has it's own velocity $\mathbf{v}$ and other fluid quantities specific to the given problem. 

\section{Density}

Density is evaluated for every particle at the beginning of the time step using the following equation called as Summation Density. 

\begin{equation}
 \rho_a = \sum_b m_b W_{ab}
\end{equation}

\noindent
where $W_{ab}$ denotes

\begin{equation}
 W_{ab} = W(\mathbf{r_{ab}}, h)
\end{equation}

\noindent
and

\begin{equation}
 \mathbf{r_{ab}} = \mathbf{r_a} - \mathbf{r_b}
\end{equation}
\noindent
where $\mathbf{r_a}$ denotes the position of particle a. The kernel typically takes the form

\begin{equation}
 W(\mathbf{r_{ab}}, h)= \frac{1}{h^N} f\left( \frac{\left|\mathbf{r_{ab}} \right|}{h}\right)
\end{equation}
\noindent
where N is the number of dimensions and the function f is typically either a Gaussian or a spline approximating a Gaussian. The smoothing length is generally between 1-1.5 times the shortes particle separation. 

\section{Equation of State}
In SPH, pressure is an explicit function of local fluid density and a quasi-incompressible equation of state is used. For this work, the isothermal EOS as follows is used.

\begin{equation}
 p_a = c_s^2 (\rho_a - \rho_0)
\end{equation}

\noindent
where $\rho_0$ is the reference density of the fluid and $c_s$ is the artificial speed of sound. Subtracting the reference density was found to lead to more accurate simulations. The reason for this is that subtracting the reference density removes a zeroth-order term associated with conservative forms of SPH pressure gradients. The speed of sound is chosen to be generally 10 times the maximum velocity of the particles. 

\section{Forces}

The forces in the simulations are divided into three parts:

\begin{itemize}
 \item Pressure forces
 \item Viscous forces
 \item Surface Tension forces
\end{itemize}

\subsection{Pressure Forces}

The pressure forces are caused due to pressure gradients. The SPH expression used to approximate the pressure gradient term in this work is:

\begin{equation}
 -\left( \frac{1}{\rho} \nabla p\right)_a = -\sum_b m_b \left( \frac{p_a + p_b}{\rho_a \rho_b}\right) \nabla_a W_{ab}
\end{equation}
\noindent
where $p_a$ is the pressure at particle a and $\nabla_a$ denotes the gradient with respect to the co-ordinates of particle a. This form of pressure gradient conserves momentum exactly, since forces acting between individual particles are antisymmetric.

\subsection{Viscous Forces}

Viscous forces were calculated using a formulation recently applied to low Reynolds Number flow. \cite{viscous} The SPH momentum equation may be written as \cite{viscous}:

\begin{equation}
 \left(\frac{d\mathbf{v_v}}{dt} \right)_a = \sum_b \frac{m_b(\nu_a + \nu_b)\mathbf{v_{ab}}}{\rho_a\rho_b} \left(\frac{1}{r_{ab}} \frac{\partial W_{ab}}{\partial r_a}\right) 
\end{equation}

\subsection{Surface Tension Forces}

\subsubsection{Calculating Interfacial Curvature}

In order to obtain the surface tension forces, the curvature $\kappa$ should be calculated. This requires calculation of surface normals and their divergence.\cite{Morris} \\

The simplest SPH expression for \textbf{n} is given by
\begin{equation}
 \mathbf{n_a} = \sum_b \frac{m_b}{\rho_b} c_b^i \nabla_a W_{ab}
\end{equation}
\noindent
where $c_b^i$ is the color index of particle b. \\
More accurace estimates of the surface normal are obtained when the color field is smoothed by convolution with the kernel. With SPH, this smoothing is done using SPH approximation of the color function.
\begin{equation}
 c_a = \sum_b \frac{m_b}{\rho_b}c_b^i W_{ab}
\end{equation}
\noindent
Additional improvements in normals is done using:
\begin{equation}
 \mathbf{n_a} = \sum_b \frac{m_b}{\rho_b} (c_b - c_a)\nabla_a W_{ab}
  \label{normal}
 \end{equation}

The simplest SPH expression for the divergence of $\mathbf{\hat n}$ is:

\begin{equation}
 \left( \nabla . \mathbf{\hat n}\right)_a = \sum_b \frac{m_b}{\rho_b} \mathbf{\hat n_b}.\nabla_a W_{ab}
\end{equation}
\noindent
A more accurate estimation of divergence is obtained using \cite{Monaghan1992}

\begin{equation}
 \left( \nabla . \mathbf{\hat n_a}\right) = \sum_b \frac{m_b}{\rho_b}(\mathbf{\hat n_b} - \mathbf{\hat n_a}). \nabla_a W_{ab} 
  \label{divergence}
 \end{equation}


If equations \ref{normal} and \ref{divergence} are used to evaluate the curvature, large errors occur at the transition region's edges. The main issue is the requirement of normalized normals $\mathbf{\hat n}$. Some distance away from the interface, \textbf{n} will be small and may have a random direction. So, any curvature using these normals would be inaccuracte. Hence, for a more accurate calculation, we used only 'reliable' normals to do the divergence calculations. The following were used to do it:

\begin{equation}
 N_a = 
 \begin{cases}
  1, & if \left|\mathbf{n_a}\right| > \epsilon \\
  0, & otherwise
 \end{cases}
  \label{reliablity}
 \end{equation}
\noindent
and

\begin{equation}
 \mathbf{\hat n_a} = 
 \begin{cases}
  \frac{\mathbf{n_a}}{\left| \mathbf{n_a} \right|}, & if N_a=1 \\
  0, & otherwise
 \end{cases}
\end{equation}

\noindent
$\epsilon$ is generally taken to be $\frac{0.01}{h}$ in this work. An intermediate estimate of curvature is done to correct \ref{divergence} for absence of some of the normals in the neighbourhood of particle a.

\begin{equation}
 \left(\nabla. \mathbf{\hat n}\right)^*_a = \sum_b min(N_a, N_b) \frac{m_b}{\rho_b}(\mathbf{\hat n_b} - \mathbf{\hat n_a}). \nabla_a W_{ab}
\end{equation}

This estimate can be corrected by a factor of $f_a$
\begin{equation}
 (\nabla . \mathbf{\hat n})_a = \frac{(\nabla . \mathbf{\hat n})_a^*}{f_a}
\end{equation}
\noindent
where
\begin{equation}
 f_a = \sum_b min(N_a, N_b) \frac{m_b}{\rho_b} W_{ab}
\end{equation}

\noindent
reflects the local number density of particles with 'reliable' normals. \\

The surface tension is then calculated as:

\begin{equation}
 (\mathbf{a_s})_a = -\frac{\sigma_a}{\rho_a}(\nabla.\mathbf{\hat n})_a \mathbf{n_a}
\end{equation}


\subsubsection{Momentum Conserving Form}
The method explained above does not guarantee exact conservation of momentum. We now discuss one method which conserves momentum. Surface tension force can be expressed as the gradient of the tensor. \cite{Surface}

\begin{equation}
 \kappa \mathbf{\hat n} \delta_s = \nabla[(\mathbf{I} - \mathbf{\hat n}\times\mathbf{\hat n})\delta_s]
\end{equation}

\noindent
The given expression can be approximated in SPH by \cite{Morris}:

\begin{equation}
 (\mathbf{a_s})_a = \left( \frac{1}{\rho} \frac{\partial S_{ij}}{\partial x_j} \right) = \sum_b m_b \frac{(S_{ij})_a + (S_{ij})_b}{\rho_a \rho_b} \nabla_{a, j} W_{ab}
\end{equation}
\noindent
where 
\begin{equation}
 S_{ij} = \delta_s(\delta_{ij} - \mathbf{\hat n_i}\mathbf{\hat n_j})
\end{equation}

\noindent
Where, $\delta_{ij}$ is the Kronecker delta, $\nabla_{a,j}W_{ab}$ is the $j^{th}$ component of the gradient of $W_{ab}$ with respect to $\mathbf{r_a}$ and the repitition of j means summation. To improve accuracy, only those normals satisfying \ref{reliablity} are used in summation. Since, the reciprocal particle summations are anti-symmetric, this conserves the momentum.\\

This method is potentially unstable as attractive forces when momentum conserving formulaitons are used are unstable when SPH particles are considered.\\
As the resolution is increased, the maximum of $\delta_s$ will increase and at some point, the method will blow up. A solution to this issue is to replace $S_{ij}$ with a modified tensor as follow:

\begin{equation}
 S_{ij}^* = S_{ij} - \delta_{ij}\times max(\delta_s)
\end{equation}


\subsubsection{Removing the Singularity}

A disadvantage of the momentum conserving formulation explained above is that the delta function introduces a singularity in the pressure field as the resolution is increased. A work around for this is the following(taking [c] = 1)

\begin{equation}
 \mathbf{F_s} = \sigma \kappa \nabla c = \nabla(\sigma \kappa c) - \sigma c \nabla \kappa
\end{equation}
\noindent
\chapter{Results and Discussions}
The following test case was implemented in PySPH using the method of Interface curvature described in the above sections. \\


\section{Stability of an Interface: Problem Statement}

Previous studies using the CSF formulation have reported a numerical instability at the fluid-fluid interface. \cite{rudman, Surface}. This instability leads to parasitic currents(non-zero velocities) at the interfaces. Using SPH, these currents could lead to particle disorder at the fluid-fluid interface, showing the effect of the diffusing at the interface. \\

This phenomena is studied here: A simple static case in a periodic domain spanning 0.5 units in x-directoin and 1 unit in y-direction with upper half with fluid of color 1 and lower half with fluid of color 0 is simulated. The fluid densities were considered constant of 1, the surface tension coeffient to be 1 and inviscid flow is considered. The speed of sound for this simulation is set to 20 untis.\cite{Morris} At t=0, a constant velocity in both directions is set so as the KE of each particle is 6 orders less than their internal energy.  The evolution of Kinetic Energy with time is recorded for different SPH parameters like smoothing length and spline. The smoothing length hdx = 1.0 and 1.5 are considered and varied smoothing length for different equations is considered where hdx = 1.0 is used all through except when calculating the surface delta function. In this way, by using a larger smoothing length for curvature calculation, more reliable estimates of curvature are obtained in the region where surface normals are non-zero.

\section{Results}

The following is the plot of KE vs time and against the results extracted from the paper \cite{Morris}. The discrepancy in the exact values is because of lack of initialization data in the paper like initial velocities' distribution, method of calculating internal energy etc., But it can be noted that the trend and the approximate values are in agreement from the plot below.


\section{Discussions}

It can be seen from the plot that the variableh case has better stability of the interface preventing much diffusion of the particles into each other.\\

It is seen that, for the same parameters, the quintic spline is more stable than the cubic spline.\\

Also, the increased smoothing length caused to a considerable amount of stability in the interface. The best stability properties are however exhibited when the smoothing length for curvature is higher than that of delta function calculation.



%****************************************************************
%                         Appendices                           
%****************************************************************
%% Additional, supporting material, such as codes, derivations, etc., can be placed in the appendix
% \appendix

\chapter{Conclusion}
Implementation of surface tension forces in the regime of Smoothed Particle Hydrodynamics(SPH) was discused in this work. Specialized expressions for curvature between two species of particles are formulated. The most straightforward method which calculates the SPH estimate of curvature and applying it directly to get surface tension force is discussed. The issue of not conserving momentum is tackled by a tensor form which might have singularities. A corrected form of pressure and surface tension forces are discussed to tackle that issue.

\chapter{Future Work}

% In the future work of this project, more test cases, benchmarks and other implementations of surface tension and multiphase flows will be done. Their extensions to three-dimensional problems in theory, which are straight-forward will be investigated and implemented. A volume density based method will be implemented for high density variation problems.

The following are to be implemented in the second phase of this project:

\begin{itemize}
 \item The already simulated problem using surface tension is to be implemented in the other two methods which were explained using the momentum conserved form and the form 
 which removes the singularities.
 \item Other benchmark problems which include Equilibrium rod, Oscillating rod \citep{Morris}, Capillary Wave, Three Phase interaction etc., \citep{Adami}
 will be implemented using different schemes.
 \item These implementations can be extended to three dimensional problems
 \item A volume density based method where the ratio of mass and density will be replaced
 by the volume of the particle. This implementation will be useful for problems with high density variations.
  \item All these equations put together can be made into a new scheme which can be readily
  used by a user in PySPH.
 \end{itemize}


%******************************************************************
%                         Bibliography or References          
%******************************************************************  
\bibliography{mylit}     

%*******************************************************************
%                         List of publications               
%******************************************************************
% %%%
\listofpublications


\noindent Put your publications from the thesis here. The packages \texttt{multibib} or \texttt{bibtopic} or \texttt{biblatex} or enumerate environment or thebibliography environment etc. can be used to handle multiple different bibliographies in the document.








%%======================================================================
%%% Local Variables: 
%%% mode: latex
%%% TeX-master: "../mainrep"
%%% End: 







            

%*******************************************************************
%                        Acknowledgements                    
%******************************************************************* 

%*******************************************************************
%                        About author                    
%*******************************************************************
% \colophon % remove this command while using this file.

% GAME OVER
%*******************************************************************
\end{document}

%%% Local Variables: 
%%% mode: latex
%%% TeX-master: t
%%% End: 
