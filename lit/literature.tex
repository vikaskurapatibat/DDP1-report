
\chapter{Basics of SPH Formulations}

In the following sections, the derivation and formulations of SPH methodology will be explained.
In SPH, we interpolate any function to be expressed in terms of its values at a set of disordered points called as the particles.
Every physical property of a fluid is regarded as a spatial function f(r). 
The ideas are given in \cite{Monaghan1977} and \cite{Lucy}. 
Each function in space be exactly represented by a convolution of the function f(r) itself with the Diract function $\delta (r)$.

\begin{equation}
 f(r) = \int f(r')\delta(r - r')dr'
\end{equation}

In SPH, the Dirac function $\delta (r)$ is replaced by a so called interpolating kernel function W(r-r', h), which resembles a Gauss distribution with compact support. Hence, in SPH, the integral approximation of any function f(r) is defined by

\begin{equation}
 F_1 (r) = \int F(r') W(r-r', h) dr' ,
\end{equation}

\noindent
where the integration is over the entire space, and W is an interpolating kernel which has the following two properties.

\begin{equation}
 \int W(r-r', h)dr' = 1
\end{equation}

\noindent
and 

\begin{equation}
 \lim_{h\to0} W(r-r', h) = \delta(r-r')
\end{equation}

\noindent
where the limited is to be interpreted as the limit of the corresponding integral interpolants. The smoothing length 'h', determines the radius of influence of the kernel function. \\

For numerical work, the integral representation is approximated by a summation interpolant as below.

\begin{equation}
 F_s (r) = \sum_{j} F_j \frac{m_j}{\rho_j} W(r-r_j, h)
 \label{interpolation}
\end{equation}

\noindent
where the summation index j is used to denote a particle label, the summation is performed over all the particles. Particle j carries mass $m_j$, position $r_j$, density $\rho_j$ and velocity $v_j$. The value of any quantity A at position $r_j$ is written as $A_j$. \\

\noindent
And the differentials and gradients are written as 

\begin{equation}
 \nabla F(r) = \sum_j \frac{m_j}{\rho_j}F_j \nabla W(r-r_j, h)
  \label{gradinterpolation}
 \end{equation}


\section{SPH formulation of basic Navier Stokes Equations}

To simulate a fluid flow, the Navier Stokes Equations are to be 
formulated into SPH. For this, equations \ref{interpolation} and \ref{gradinterpolation}
are used. These give us the approximations for the differentials of density, velocity and energy.

\subsection{Continuity Equation}

Continuity Equation gives the evolution of density. The formulations
are as given below:

\begin{equation}
 \rho_i = \sum_j m_j W_{ij}
 \label{summation}
\end{equation}

This form of equation for density is called the Summation Density
which uses SPH formulation of \ref{interpolation}. \\

\noindent
Density can also be evolved by using $\frac{d\rho}{dt} + \nabla . \mathbf{v} = 0$

\begin{equation}
 \frac{d\rho_i}{dt} = \sum_j m_j \mathbf{v_{ij}} \nabla_i W_{ij}
  \label{Continuity}
 \end{equation}

where $\mathbf{v_{ij}} = \mathbf{v_i} - \mathbf{v_j}$.\\


\subsection{Momentum Equation}

The momentum Equation generally contains pressure and viscous terms which are discussed here.
Any extra forces like stress, surface tension forces are to be dealth
seperately and added to the momentum equation.

\subsubsection{Pressure Force}

The force per unit mass due to pressure is given as -$\frac{\nabla P}{\rho}$.
The gradient of the pressure can be estimated as:

\begin{equation}
 \rho_i \nabla P_i = \sum_j m_j (P_j - P_i) \nabla_i W_{ij}
\end{equation}

This equation gives zero forces for constant pressures but doesn't
conserve linear and angular momentum. Hence, a differen SPH formulation
is required.\\

For the same reason, a different formulation of the pressure gradient is used as follows:

\begin{equation}
 \frac{\nabla P}{\rho} = \nabla \left( \frac{P}{\rho}\right) + \frac{P}{\rho^2}\nabla P
\end{equation}

This when written in SPH form gives pressure to be as:

\begin{equation}
 \frac{d\mathbf{v_i^p}}{dt} = -\sum_j m_j \left( \frac{P_j}{\rho_j^2} + \frac{P_i}{\rho_i^2}\right)\nabla_i W_{ij}
\end{equation}

\noindent
This form of pressure force conserves momentum as it is asymmetric. 

\subsubsection{Viscous Force}

If there is no Physical Laminar viscosity in the flow, in general
an artificial viscosity is used in SPH to improve the numerical stability.\\

The commonly used artificial viscosity used term as given in \cite{Monaghan1983}:

\begin{equation}
 \frac{d\mathbf{v_i^\nu}}{dt} = - \sum_j m_j \Pi_{ij} \nabla_i W_{ij}
\end{equation}

where the viscosity term is:

\begin{equation}
 \Pi_{ij} = -\nu \left( \frac{min(\mathbf{v_{ij}}, \mathbf{x_{ij}}, 0)}{(\mathbf{x_{ij}}^2 + \epsilon h_{ij}^2)}\right), 
  \label{pi}
 \end{equation}

\noindent
where the viscous factor is:

\begin{equation}
 \nu = \frac{\alpha h_{ij}c_s}{\rho_{ij}}
\end{equation}

\noindent
where $\alpha$ is the viscosity constant, $c_s$ is the artificial
speed of sound, $\mathbf{v_{ij}} = \mathbf{v_i} - \mathbf{v_j}$
and $\mathbf{x_{ij}} = \mathbf{x_i} - \mathbf{x_j}$. In equation \ref{pi},
$\epsilon \sim 0.01$ is used to avoid blowing up when $x_{ij} = 0$ and average smoothing length is given by
$h_{ij} = 0.5*(h_i + h_j)$. This viscous force basically gives a repulsive force between approaching particles.

\subsection{The Energy Equation}

The equation for the rate of change of thermal energy per unit mass is given by:

\begin{equation}
 \frac{du}{dt} = -\left( \frac{P}{\rho} \right)\nabla. \mathbf{v}
\end{equation}

\noindent
which when written in SPH form gives:

\begin{equation}
 \frac{du_i}{dt} = \left(\frac{P_i}{\rho_i^2}\right)\sum_j m_j \mathbf{v_{ij}}.\nabla_iW_{ij}
\end{equation}

\noindent
or by transforming the equation to:

\begin{equation}
 \frac{du}{dt} = -\nabla\left(\frac{P\mathbf{v}}{\rho}\right) + \mathbf{v}.\nabla\left(\frac{P}{\rho}\right)
\end{equation}

\noindent 
the equation can be rewritten as:

\begin{equation}
 \frac{du_i}{dt} = \sum_j \left( \frac{P_j}{\rho_j^2} \right)\mathbf{v_ij}.\nabla_i W_{ij}
\end{equation}

\noindent
By taking the average of the above two representations, we find:

\begin{equation}
 \frac{du_i}{dt} = \frac{1}{2} \sum_j m_j \left( \frac{P_j}{\rho_j^2} + \frac{P_i}{\rho_j^2}\right)\mathbf{v_{ij}}.\nabla_i W_{ij}
\end{equation}

\noindent
note that this has symmetric factors.


\section{Moving the Particles}

There are two ways to move the particles. One way is:

\begin{equation}
 \frac{d\mathbf{r_i}}{dt} = \mathbf{v_i}
\end{equation}

\noindent
or the XSPH variant as given in \cite{Monaghan1983}:

\begin{equation}
\frac{d\mathbf{r_i}}{dt} = \hat {\mathbf{v_i}} = \mathbf{v_i} + \epsilon \sum_j m_j \left(\frac{\mathbf{v_{ji}}}{\rho_{ij}}\right) W_{ij}
\end{equation}

\noindent
where $\rho_{ij} = 0.5*(\rho_i + \rho_j)$ and $\epsilon (0 \leq \epsilon \leq 1)$ as a constant. 
This XSPH correction helps in moving a particles with a velocity that is closer to the
average velocity in its neighbourhood. XSPH is found to increase dispersion instead of introducin
dissipation. The XSPH correction is useful in simulating nearly incompressible fluids like water, which keeps 
particles' motion orderly in the absence of viscosity.

\section{Equation of State}

For a closed system of equations, we need an Equation of State(EOS) which evaluates pressure.
Generally, an equation of state provides a direct coupling of density and pressure. An incompressible fluid 
can be thought of as a weakly compressible fluid. This can be done by using a stiff EOS.
Hence, pressure variations are high for small variations of density also. A numerical speed of sound
$c_0 = \sqrt{\left(\frac{dP}{d\rho}\right)_s}$ is used in the EOS.\\

The Suitable equation as given by Tait is given by :

\begin{equation}
 \Delta P = \frac{c_0^2 \rho_0}{\gamma} \left[ \left( \frac{\rho}{\rho_0} \right)^{\gamma} - 1 \right]
\end{equation}

\noindent
Using this equation, the pressure can be computed and used in the momentum equations.

