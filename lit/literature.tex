
\chapter{SPH Formulations}

In the following sections, the derivation and formulations of SPH methodology will be explained. In SPH, we interpolate any function to be expressed in terms of its values at a set of disordered points called as the particles. Every physical property of a fluid is regarded as a spatial function f(r). The ideas are given in \cite{Monaghan1977}. Each function in space be exactly represented by a convolution of the function f(r) itself with the Diract function $\delta (r)$.

\begin{equation}
 f(r) = \int f(r')\delta(r - r')dr'
\end{equation}

In SPH, the Dirac function $\delta (r)$ is replaced by a so called interpolating kernel function W(r-r', h), which resembles a Gauss distribution with compact support. Hence, in SPH, the integral approximation of any function f(r) is defined by

\begin{equation}
 F_1 (r) = \int F(r') W(r-r', h) dr' ,
\end{equation}

\noindent
where the integration is over the entire space, and W is an interpolating kernel which has the following two properties.

\begin{equation}
 \int W(r-r', h)dr' = 1
\end{equation}

\noindent
and 

\begin{equation}
 \lim_{h\to0} W(r-r', h) = \delta(r-r')
\end{equation}

\noindent
where the limited is to be interpreted as the limit of the corresponding integral interpolants. The smoothing length 'h', determines the radius of influence of the kernel function. \\

For numerical work, the integral representation is approximated by a summation interpolant as below.

\begin{equation}
 F_s (r) = \sum_{j} F_j \frac{m_j}{\rho_j} W(r-r_j, h) ,
\end{equation}

\noindent
where the summation index j is used to denote a particle label, the summation is performed over all the particles. Particle j carries mass $m_j$, position $r_j$, density $\rho_j$ and velocity $v_j$. The value of any quantity A at position $r_j$ is written as $A_j$. \\

\noindent
And the differentials and gradients are written as 

\begin{equation}
 \nabla F(r) = \sum_j \frac{m_j}{\rho_j}F_j \nabla W(r-r_j, h)
\end{equation}

